\documentclass{article}

% preamble
\usepackage{mathtools}
\usepackage{hyperref}
\DeclareMathOperator{\rank}{rank}
\DeclareMathOperator{\lcm}{lcm}

\begin{document}

% Remember that LaTeX doesn't like blank lines within math environments
% and that the resulting error messages are not helpful.

\section*{Computational tricks for tied score detection in global pairwise sequence alignment algorithms}

Detecting ties in global pairwise sequence alignments is not an
interesting or important problem, but here are some notes
regarding some computational considerations.

The computational concerns about equality detection
are completely uninteresting from a mathematics research perspective.
Even from an applied computational perspective, equality detection for the
kinds of expressions mentioned in this note has already been
implemented by widely available computer algebra systems.

All schemes considered here are assumed to score a partial
pairwise sequence alignment according to a finite dimensional vector
of `move type' counts.
The finite set of `move types' depends on details of the scoring algorithm.
For example the set of move types for a basic scheme could be
\{gap, mismatch, match\},
or for more complicated schemes it could be more exotic like
\{Seq1GapInit, Seq1GapExtend, matchAA, mismatchAG, \dots\}.

In this framework the sufficiency of the count vector for computing the score
immediately gives a condition indicating a tie:
if two partial alignments share the same count vector
then they must also share the same score.
On the other hand, two different count vectors may
sometimes also share the same score --
this mathematical coincidence is annoying if you are trying to
determine which partial alignment should be preferred,
and the following notes attempt to explain the nature of these
coincidences and how they can be detected efficiently within the
context of a dynamic programming alignment algorithm.

\section*{Examples and analyses under various scoring schemes}

The following examples assume that we have a scoring function with
a set of $k = 3$ move types and that we want to evaluate a partial alignment
whose move type count vector is $n = (n_1, n_2, n_3)^T$.

\subsection*{Additive integer scores}

In this simple scoring scheme,
move type $i$ is associated with an integer value $a_i$
and the partial alignment score is $\sum_i^k a_i n_i$.
Tie detection is simple in this case
because you can compare the integer scores directly.

As an example, let the vector of move type values be
$a = (1, 2, 3)^T$.
If we let two different partial alignments have move type counts
$u = (1, 1, 0)^T$ and $v = (0, 0, 1)^T$,
then the partial alignments both have score $3$.
Similarly, if two different partial alignments have move type count vectors
$u = (5, 0, 0)^T$ and $v = (0, 1, 1)^T$
then the partial alignments will both have score $5$.

\subsection*{Multiplicative rational number scores}

This scoring scheme is nearly as simple.
If move type $i$ is associated with a rational number $b_i$
then the partial alignment score for move type count vector $n$ is
$\prod_i^k {b_i}^{n_i}$.
The logarithm of the score is $\sum_i^k n_i \log{b_i}$.
As an example, let $b = (10/3, 4/5, 8/3)^T$
be the vector of move type values.
Continuing the example,
if two different partial alignments have move type counts
$u = (1, 1, 0)^T$ and $v = (0, 0, 1)^T$,
then the partial alignments both have score $8/3$.
On the other hand,
if two different partial alignments have move type count vectors
$u = (5, 0, 0)^T$ and $v = (0, 1, 1)^T$,
then the partial alignment scores will differ --
the score of $u$ would be $(10/3)^5 = 100000/243$
and the score of $v$ would be $(4/5)*(8/3) = 32/15$.

As was the case for the additive scoring scheme,
the rational number scores could also be directly compared.
If a naive internal representation of these scores is used,
for example storing the numerator and denominator of the reduced
fraction using base-2 integers, then the comparison time
and storage requirement will grow at least linearly with the
length of the partial alignment, which is undesirably slow.

It is possible to do better by representing and comparing the partial scores
in 'log space'.
We want to enable exact score comparisons,
so we do not use floating point representations which
approximate rational numbers as integer multiples of powers of 2.
Instead, the partial score can be represented by a vector
of integer powers of pairwise coprime factors appearing in the numerators
and denominators of $b$.
Continuing the example above, the factors $\{2, 3, 5\}$ could be used,
and the entrywise logarithm of $b$ could be written as
\begin{equation*}
\begin{pmatrix*}
\log{b_1} \\
\log{b_2} \\
\log{b_3}
\end{pmatrix*}
=
\begin{pmatrix*}[r]
1 & -1 & 1 \\
2 & 0 & -1 \\
3 & -1 & 0
\end{pmatrix*}
\begin{pmatrix*}
\log{2} \\
\log{3} \\
\log{5}
\end{pmatrix*}.
\end{equation*}
and the score would be
\begin{equation*}
\exp{\left(
\begin{pmatrix*}
n_1 & n_2 & n_3
\end{pmatrix*}
\begin{pmatrix*}[r]
1 & -1 & 1 \\
2 & 0 & -1 \\
3 & -1 & 0
\end{pmatrix*}
\begin{pmatrix*}
\log{2} \\
\log{3} \\
\log{5}
\end{pmatrix*}
\right)}.
\end{equation*}
%
Note that the score depends on the integer count vector $n$
only through the transformed vector $n^T G$ where
\begin{equation*}
G =
\begin{pmatrix*}[r]
1 & -1 & 1 \\
2 & 0 & -1 \\
3 & -1 & 0
\end{pmatrix*}.
\end{equation*}
%
Because $\{2, 3, 5\}$ are pairwise coprime,
two count vectors will give the same scores
if and only if the transformed vectors are the same.
In the first example above with count vectors
$u = (1, 1, 0)^T$ and $v = (0, 0, 1)^T$,
the transformed vectors $u^T G$ and $v^T G$ are both $(3, -1, 0)^T$.
In the second example above with count vectors
$u = (5, 0, 0)^T$ and $v = (0, 1, 1)^T$,
the transformed vectors are $u^T G = (5, -5, 5)$ and $v^T G = (5, -1, -1)$.

In less abstract terms,
the idea is to track a vector of integer powers of pairwise coprime factors
instead of tracking the vector of move type counts.
Instead of incrementing the $i$th entry of
the count vector when move type $i$ is encountered,
row $i$ of $G$ is added (in the sense of vector addition)
to the vector that tracks the integer powers of factors.
For detecting tied scores,
the benefit of tracking integer powers instead of move type counts
is that two power vectors are identical if and only if the scores
are identical, whereas it is possible (as seen in the example above)
that two move type count vectors may be different from each other while
still sharing the same score evaluation.

\subsubsection*{a digression on optional computational tricks}

In closing this section on multiplicative rational number scores,
I'll mention a couple
small computational tricks that can be used in combination
with the idea of transforming the count vector.

The first trick involves noticing that although computing the prime
factorization of the numerators and denominators of $b$ may be the most
obvious way to compute a pairwise coprime factorization,
it is not always necessary.
Any pairwise coprime factorization is sufficent,
and this factorization can be computed by ``factor refinement"
more efficiently than by computing the prime factorization.
As an extreme example of the efficiency improvement,
if someone concocts a scoring scheme that involves a large integer like
\href{https://en.wikipedia.org/wiki/RSA\_numbers#RSA-1024}{RSA-1024}
then a method based on prime factorization will fall down that
rabbit hole whereas a method based on factor refinement would
compute a coprime factorization without trying to solve the RSA challenge.

The second trick involves decomposing $G$
to yield a transformation with the desirable
properties mentioned above but without having to track as many integers.
The number of entries in the transformed vector explained earlier
in this section is equal to the number of columns of $G$,
but linear algebra tricks can be used to find a different
transformation with the same desirable properties and which
tracks a vector with only $\rank{(G)}$ entries.
This trick involves a matrix decomposition of $G$ which
needs to be computed only once.
For the example in this section,
a vector that tracks only $\rank{(G)} = 2$ integers instead of 3 integers
could be used, and it would involve a decomposition like
\begin{equation*}
G =
\begin{pmatrix*}[r]
1 & -1 \\
2 & -1 \\
3 & -2
\end{pmatrix*}
\begin{pmatrix*}[r]
1 & 1 & -2 \\
0 & 2 & -3
\end{pmatrix*}
=
\begin{pmatrix*}[r]
1 & -1 & 1 \\
2 & 0 & -1 \\
3 & -1 & 0
\end{pmatrix*}.
\end{equation*}


\subsection*{Multiplicative univariate rational function scores}

The algebraic structure of this scoring scheme
is similar to that of multiplicative rational number scoring.
If move type $i$ is associated with a univariate
rational function $c_i$
then the partial alignment score for move type count vector $n$ is
$\prod_i^k {c_i}^{n_i}$.
We will reuse the structure of the example from the previous section,
but instead of prime integer factors $\{2, 3, 5\}$ we will use
irreducible univariate polynomial factors $\{x, x-1, x+1\}$.
Following that analogy, for this example we have
\begin{equation*}
c =
\left(
\frac{x \left( x + 1 \right)}{x-1},
\frac{x^2}{x+1},
\frac{x^3}{x-1}
\right)^T.
\end{equation*}
%
Because the structure is like that of multiplicative rational number
scoring I think that similar tricks apply,
but a practical complication is that software library support
for rational functions is less readily available than for rational numbers.
Just for comparison with the previous section,
here is the log space form:
%
\begin{equation*}
\begin{pmatrix*}
\log{c_1} \\
\log{c_2} \\
\log{c_3}
\end{pmatrix*}
=
\begin{pmatrix*}[r]
1 & -1 & 1 \\
2 & 0 & -1 \\
3 & -1 & 0
\end{pmatrix*}
\begin{pmatrix*}
\log{x} \\
\log{\left(x - 1\right)} \\
\log{\left(x + 1\right)}
\end{pmatrix*}
\end{equation*}
%
The scoring method in this section may seem strange
because the score is a function of an apparently useless placeholder
value $x$, but nevertheless it is meaningful to detect equality of two
rational functions.
In the next section we look at what happens
when a particular value of $x$ is chosen,
and we see that for a certain class of choices the
equality of two rational functions
is equivalent to the equality of the evaluations of the functions
at that point.


\subsection*{Multiplicative univariate rational functions
evaluated at transcendental points}

It may be clear that comparing two rational functions for equality
is not equivalent to comparing the equality of the evaluations
of the two rational functions at a given point.
For example $x$ and $x^2$ are distinct polynomials with integer coefficients
but they both evaluate to $1$ at the point $x=1$.

On the other hand, this kind of coincidence cannot occur
when rational functions are evaluated at a transcendental point.
Therefore if the scoring scheme specifies that the
multiplicative univariate rational functions are to be evaluated
only at transcendental points, then any technique for detecting
equality of rational functions `goes through' for detecting
equality of transcendental evaluations of those functions.

Continuing the example from the previous section,
if we let $x$ take the transcendental value $e^{1/5}$
then we have
%
\begin{equation*}
\begin{pmatrix*}
c_1 \\
c_2 \\
c_3
\end{pmatrix*}
=
\begin{pmatrix*}
\frac{e^{1/5} \left( e^{1/5} + 1 \right)}{e^{1/5} - 1} \\
\frac{e^{2/5}}{e^{1/5} + 1} \\
\frac{e^{3/5}}{e^{1/5} - 1}
\end{pmatrix*}.
\end{equation*}
%
Even for such complicated-looking scoring schemes,
I think that with a suitable library for manipulating
(large sparse) univariate polynomials with integer coefficients
it would be possible to
precompute a basis that allows an integer vector to be
tracked efficiently in a way that makes tied score detection
equivalent to detecting equality of integer vectors.
This particular example would not require working with large
sparse polynomials,
but I have in mind an application that would require
working with polynomials of degree $\lcm{\{23^2, 24^2, 26^2, 27^2\}}$
and with fewer than a dozen nonzero coefficients.

Without such an implementation,
univariate rational functions can be partially supported
by treating their numerators and denominators
analogously to prime integers within the framework of
multiplicative rational number scoring.
This gives an incomplete tie detection method whose ignorance
of polynomial factors shared between polynomials
may cause it to miss ties.
But maybe this detection method would miss fewer ties
than the method that directly compares move type count vectors.

\end{document}
