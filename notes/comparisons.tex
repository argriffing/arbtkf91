\documentclass{article}
\usepackage{mathtools}
\usepackage{blkarray}
%\usepackage{kbordermatrix}

% This is not available for 2015 debian/ubuntu texlive.
%\fboxsep0pt
%\left\{\fbox{$\kbordermatrix{&1&2\\
	%1&A&B\\
	%2&C&D\\
	%3&E&F}$}\right\}

%% right alignment of columns
%\begin{matrix*}[r]
%2 & 0 & -1 \\
%1 & -1 & 1 \\
%3 & -1 & 0
%\end{matrix*}


\begin{document}

% Remember that LaTeX doesn't like blank lines within math environments
% and that the resulting error messages are not helpful.

\section*{additive integer scores}

%\begin{equation*}
%\begin{blockarray}{ccc}
%%2 & 3 & 5 \\
%\begin{block}{[ccc]}
%2 & 0 & -1 \\
%1 & -1 & 1 \\
%3 & -1 & 0 \\
%\end{block}
%\end{blockarray}
%\end{equation*}

%\begin{equation*}
%\bordermatrix{& & & \cr
	%& 2 & 0 & -1 \cr
	%& 1 & -1 & 1 \cr
	%& 3 & -1 & 0 \cr}
%\end{equation*}

\begin{equation*}
\begin{pmatrix*}
a_1 \\
a_2 \\
a_3
\end{pmatrix*}
=
\begin{pmatrix*}
1 \\
2 \\
3
\end{pmatrix*}
\end{equation*}


\section*{multiplicative rational scores}

%\begin{equation*}
%\begin{blockarray}{ccc}
%2 & 3 & 5 \\
%\begin{block}{[ccc]}
%2 & 0 & -1 \\
%1 & -1 & 1 \\
%3 & -1 & 0 \\
%\end{block}
%\end{blockarray}
%\end{equation*}

%\begin{equation*}
%\bordermatrix{& 2 & 3 & 5 \cr
	%& 2 & 0 & -1 \cr
	%& 1 & -1 & 1 \cr
	%& 3 & -1 & 0 \cr}
%\end{equation*}

\begin{equation*}
\begin{pmatrix*}
\log{b_1} \\
\log{b_2} \\
\log{b_3}
\end{pmatrix*}
=
\begin{pmatrix*}[r]
1 & -1 & 1 \\
2 & 0 & -1 \\
3 & -1 & 0
\end{pmatrix*}
\begin{pmatrix*}
\log{2} \\
\log{3} \\
\log{5}
\end{pmatrix*}
\end{equation*}



\section*{multiplicative univariate rational function scores}

%\bordermatrix{& x & x-1 & x+1 \cr
	%& 2 & 0 & -1 \cr
	%& 1 & -1 & 1 \cr
	%& 3 & -1 & 0 \cr}


\begin{equation*}
\begin{pmatrix*}
\log{c_1} \\
\log{c_2} \\
\log{c_3}
\end{pmatrix*}
=
\begin{pmatrix*}[r]
1 & -1 & 1 \\
2 & 0 & -1 \\
3 & -1 & 0
\end{pmatrix*}
\begin{pmatrix*}
\log{x} \\
\log{\left(x - 1\right)} \\
\log{\left(x + 1\right)}
\end{pmatrix*}
\end{equation*}

\begin{equation*}
\begin{pmatrix*}
c_1 \\
c_2 \\
c_3
\end{pmatrix*}
=
\begin{pmatrix*}
\frac{x \left( x + 1 \right)}{x-1} \\
\frac{x^2}{x+1} \\
\frac{x^3}{x-1}
\end{pmatrix*}
\end{equation*}


\end{document}
